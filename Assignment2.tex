\documentclass{report}
% Change "article" to "report" to get rid of page number on title page
\usepackage{amsmath,amsfonts,amsthm,amssymb}
\usepackage{setspace}
\usepackage{Tabbing}
\usepackage{fancyhdr}
\usepackage{lastpage}
\usepackage{extramarks}
\usepackage{chngpage}
\usepackage{soul,color}
\usepackage{listings}
\usepackage{enumerate}
\usepackage{graphicx,float,wrapfig}
\usepackage{pifont}
\usepackage{graphicx}
\usepackage[english]{babel}
\usepackage{tikz}
\usepackage[]{algorithm2e}
% In case you need to adjust margins:
\topmargin=-0.45in      %
\evensidemargin=0in     %
\oddsidemargin=0in      %
\textwidth=6.5in        %
\textheight=9.0in       %
\headsep=0.25in         %

\title{Assignment 2 - Comp 632 - Machine Learning}

% Homework Specific Information
\newcommand{\hmwkTitle}{Assignment 2}                     % Adjust this
\newcommand{\hmwkDueDate}{Wednesday, February 18 2015}                           % Adjust this
\newcommand{\hmwkClass}{COMP 632}


\newcommand{\hmwkClassInstructor}{Dr. Doina Precup}
\newcommand{\hmwkAuthorName}{Geoffrey Stanley}
\newcommand{\hmwkAuthorNumber}{260645907}
\newcommand{\Pp}{\mathbb{P}}
\newcommand{\Ev}{\mathbb{E}}
\newcommand{\cov}{\text{Cov}}
\newcommand{\Z}{\mathbb{Z}}
\newcommand{\R}{\mathbb{R}}
\newcommand{\dd}{\, \mathrm{d}}

% Setup the header and footer
\pagestyle{fancy}                                                       %
\lhead{\hmwkAuthorName}                              %
\chead{}
\rhead{\hmwkClass: \hmwkTitle}                                          %

\lfoot{}
\cfoot{}                                                                %
\rfoot{Page\ \thepage\ of\ \pageref{LastPage}}                          %
\renewcommand\headrulewidth{0.4pt}                                      %
\renewcommand\footrulewidth{0.4pt}                                      %

% This is used to trace down (pin point) problems
% in latexing a document:
%\tracingall
\definecolor{mygreen}{rgb}{0,0.6,0}
\lstset{commentstyle=\color{mygreen}, frame=single,  language=R, showspaces=false, showstringspaces=false}

%%%%%%%%%%%%%%%%%%%%%%%%%%%%%%%%%%%%%%%%%%%%%%%%%%%%%%%%%%%%%
% Make title
\title{\vspace{2in}\textmd{\textbf{\hmwkClass:\ \hmwkTitle}}\\
\normalsize\vspace{0.1in}\small{Due\ on\ \hmwkDueDate}\\
\vspace{0.1in}\large{\textit{Presented to \hmwkClassInstructor}}\vspace{3in}}
\date{}
\author{\textbf{\hmwkAuthorName}\\
    \textbf{Student ID: \hmwkAuthorNumber}}
%%%%%%%%%%%%%%%%%%%%%%%%%%%%%%%%%%%%%%%%%%%%%%%%%%%%%%%%%%%%%

\begin{document}
\maketitle
\section*{Question 1}
\subsection*{A)}
For a function to be considered a kernel function the kernel matrix defined as
$K_{ij}=K(x_i,x_j)$ must have two properties:
\begin{enumerate}
  \item be symmetric
  \item be positive semidefinite
\end{enumerate}
As such, a Kernel matrix must abide by the following:
\begin{equation}
  K_{ij} = K_{ji}
\end{equation}
\begin{equation}
  z^{T}Kz\geq 0
\end{equation}
Where $z$ is an arbitrary vector.
\subsection*{B)}
As $l$ increases words will have a tendency of having higher scores when compared
with itself then any other words. This will result in a diagonal matrix.
\subsection*{C)}
Yes.
\subsection*{D)}

\section*{Question 2}
\subsection*{A)}
\includegraphics[width=175px, keepaspectratio]{3points.jpg}
\subsection*{B)}
The VC-dimension of this hypthesis class is 4. This is because it can successfully
shatter all configurations of 4 points. However, it would not be able to do so for all
configuration of 5 points.
\subsection*{C)}
The VC-dimension of any type of boolean combination of 2 linear classifiers is
also 4.

\section*{Question 3}
\subsection*{A)}
Given the log-likelihood of a hypothesis $h$ :

\begin{equation}
  \log L(h) = \sum_{i=1}^{m} \log P(y_i|x_i,h)
\end{equation}

And the probability of an example x belonging to class K as being :

\begin{equation}
  P(K | x) = 1 - \sum_{i=1}^{K-1}h^i(x)
\end{equation}

We can derive the log likelihood for a set of hypotheses and a given data set D as:

\begin{equation}
  \log L(h) = \sum_{i=1}^{m} \sum_{j=1}^{K} \log \left( 1 - \sum_{l=1}^{K-1}h^l(x_i) \right)
\end{equation}


\subsection*{B)}
\subsection*{C)}

\section*{Question 4}
\subsection*{A)}
\subsection*{B)}
\subsection*{C)}
\subsection*{D)}
\subsection*{E)}

\end{document}
